\begin{thebibliography}{9}

\bibitem{perna2019}
Perna, D., & Tagarelli, A. (2019). 
\textit{Deep auscultation: Predicting respiratory anomalies and diseases via recurrent neural networks}. 
Proceedings of the 2019 IEEE International Conference on Bioinformatics and Biomedicine (BIBM), 2821–2827.

\bibitem{demir2021}
Demir, F., & Sengur, A. (2021).
\textit{A deep CNN-LSTM model for classification of lung sounds}.
Biomedical Signal Processing and Control, 70, 102896.

\bibitem{ibrahim2020}
Ibrahim, A., & Hussain, A. (2020).
\textit{Lung sound classification using recurrent neural networks and short-time Fourier transform}.
Computers in Biology and Medicine, 122, 103771.

\bibitem{aygun2022}
Aygun, M., & Sert, E. (2022).
\textit{Respiratory sound classification using EfficientNet and spectrogram augmentation}.
IEEE Access, 10, 45329–45340.

\bibitem{icbhi2017}
Rocha, B. M., Filos, D., Mendes, L., Vogiatzis, I., Perantoni, E., Kaimakamis, E., ... & Nascimento, J. C. (2017).
\textit{An open access database for the evaluation of respiratory sound classification algorithms}.
Physiological Measurement, 38(9), N1–N13.

\bibitem{hsu2023clap}
Wei-Hung Hsu, Hung-Yi Lee, Po-Han Yang, Arsha Nagrani, Joel Shor, Luciana Ferrer, and Ignacio Lopez Moreno. 
\newblock CLAP: Learning Audio-Text Joint Embeddings from Large-Scale Webly-Supervised Data.
\newblock \emph{arXiv preprint arXiv:2302.03102}, 2023.

\end{thebibliography}
