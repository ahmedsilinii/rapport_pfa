
\section{Introduction}
Respiratory diseases are among the leading causes of morbidity and mortality worldwide. Conditions such as asthma, pneumonia, bronchitis, and chronic obstructive pulmonary disease (COPD) often present with distinct acoustic signatures in lung sounds. Early and accurate detection of these anomalies can significantly improve clinical outcomes, reduce hospital admissions, and lower healthcare costs. Traditional auscultation with a stethoscope, though effective, is highly subjective and dependent on clinical expertise. This project leverages deep learning and multimodal processing to automate the classification of respiratory sounds using both audio and clinical metadata, aiming to support clinicians with an objective and reproducible decision-support tool.
\newpage

\section*{2.1 Project Domain – Respiratory Anomalies}

\paragraph{2.1.1 Respiratory Anomalies and Their Significance\\}
Respiratory anomalies are abnormal sounds heard during lung auscultation, which are indicative of underlying pulmonary conditions. Common anomalies include:

\begin{itemize}
    \item \textbf{Wheezing:} A high-pitched, musical sound typically associated with narrowed airways and conditions such as asthma or COPD.
    \item \textbf{Crackles (Rales):} Discontinuous popping sounds heard during inhalation, often indicating fluid in the alveoli as seen in pneumonia or pulmonary fibrosis.
\end{itemize}

\paragraph{2.1.2 Importance of Early Detection in Healthcare\\\\}
Early diagnosis of respiratory diseases through anomaly detection can:

\begin{itemize}
    \item Enable timely medical intervention and prevent disease progression.
    \item Reduce the need for costly imaging or invasive diagnostic procedures.
    \item Facilitate remote diagnosis in telemedicine and resource-limited settings.
    \item Assist in large-scale screening, especially during pandemics or respiratory outbreaks.
\end{itemize}

This project's integration of both audio recordings and patient metadata (extracted from clinical PDF reports) aims to emulate real-world diagnostic scenarios where doctors rely on both auscultatory findings and patient history for diagnosis.
