\section*{Conclusion}
\addcontentsline{toc}{section}{Conclusion}

\label{sec:chapter2_conclusion}

This chapter introduced the key domains that underpin the design of our AI-powered respiratory diagnosis system. These domains—medical anomalies, deep learning, and mobile development—serve as the conceptual foundation for the project's implementation.

First, we explored the nature of respiratory anomalies and their clinical significance, highlighting sounds such as wheezing, crackles, and stridor. These sounds provide early indicators of diseases like asthma, pneumonia, and COPD. Understanding their acoustic characteristics is essential for developing diagnostic algorithms that can detect them automatically.

Second, we reviewed core deep learning concepts, with a focus on audio classification using models such as CLAP, as well as the use of large language models (LLMs) for natural language understanding. These AI tools enable the system to both analyze user-recorded audio and provide conversational explanations and guidance through the integrated assistant.

Finally, we presented key mobile development concepts relevant to our implementation. Cross-platform development, modular architecture, responsive design, and data privacy principles ensure that the system remains accessible, secure, and user-friendly across devices. These concepts support the delivery of accurate, real-time diagnostics directly through a mobile interface.

Together, these foundations—clinical, algorithmic, and technical—set the stage for the practical implementation of our system, detailed in the chapters that follow.
