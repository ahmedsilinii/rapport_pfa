\section{Deployment (Model Serving)}

\subsection{Overview}
To enable real-time inference, the trained multimodal CLAP-based classifier was deployed using a FastAPI web server. This server exposes a RESTful endpoint allowing users to upload a WAV audio file along with a patient information file in PDF format. The server extracts text from the PDF and combines it with the audio input to produce a diagnostic prediction.

\subsection{File Structure}
The deployment folder contains the following core components:

\begin{itemize}
    \item \texttt{main.py} – FastAPI application containing preprocessing, inference, and routing logic.
    \item \texttt{model.pt} – Trained PyTorch classifier model weights.
    \item \texttt{label\_encoder.pkl} – Serialized \texttt{LabelEncoder} used to decode class indices into labels.
    \item \texttt{requirements.txt} – Specifies required Python libraries including \texttt{torch}, \texttt{transformers}, \texttt{PyMuPDF}, and \texttt{fastapi}.
    \item \texttt{Dockerfile} – Enables containerization of the app for consistent and scalable deployment.
\end{itemize}

\subsection{API Endpoints}
Two endpoints are exposed:
\begin{itemize}
    \item \texttt{GET /} – Basic health check returning a simple greeting.
    \item \texttt{POST /predict} – Accepts:
    \begin{itemize}
        \item A WAV audio file.
        \item A PDF file containing patient information (e.g., age, sex, symptoms).
    \end{itemize}
    and returns a JSON response containing the predicted diagnostic label.
\end{itemize}

\subsection{Processing Pipeline}
Upon receiving a request:
\begin{enumerate}
    \item The audio file is validated, loaded, resampled to 48\,kHz, converted to mono, and trimmed to a maximum of 8 seconds.
    \item The PDF file is parsed using \texttt{PyMuPDF} (\texttt{fitz}) to extract text-based metadata (e.g., "Patient is Male 36 , with BMI...").
    \item Audio and extracted text are processed using the CLAP processor to generate embeddings.
    \item Embeddings are passed to the trained classifier to generate class probabilities.
    \item The most probable class is decoded using \texttt{LabelEncoder} and returned.
\end{enumerate}

\subsection{Robustness and Error Handling}
The API includes robust validation and fallback strategies:
\begin{itemize}
    \item Rejects non-WAV or non-PDF inputs.
    \item Handles corrupted or overly long audio clips.
    \item Gracefully processes empty or unreadable PDFs.
    \item Returns descriptive error messages and HTTP status codes for all edge cases.
\end{itemize}

\subsection*{Conclusion}
This section outlined the full CRISP-DM pipeline used to develop a multimodal respiratory sound classifier. By combining lung audio with patient metadata extracted from PDFs, the system leveraged CLAP embeddings for accurate, context-aware predictions. The final model was deployed using FastAPI, enabling real-time and scalable inference suitable for clinical settings.
