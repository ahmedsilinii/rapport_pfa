% 5.1 Business Understanding.tex
\label{subsec:business-understanding}

The AI assistant, built using a Large Language Model (LLM) enhanced with Retrieval-Augmented Generation (RAG), is designed to provide users with accurate, contextual information about respiratory symptoms and diseases through natural language interaction. This tool complements the audio classification system by enabling users to ask questions and receive clear explanations and guidance related to their respiratory health.

\subsection*{Project Goals}
\begin{itemize}
    \item \textbf{Deliver Clear Medical Information:} Provide understandable explanations about respiratory anomalies such as wheezing and crackles detected by the system.
    
    \item \textbf{Support Decision-Making:} Assist both patients and doctors in interpreting symptoms and accessing relevant medical knowledge to make better-informed decisions.
    
    \item \textbf{Enhance Accessibility:} Offer users immediate and easy access to reliable respiratory health information without the need for expert presence.
    
    \item \textbf{Encourage User Engagement:} Foster continuous interaction by allowing users to ask follow-up questions and explore related health topics.
    
    \item \textbf{Integrate Seamlessly:} Work alongside the audio classification module to create a comprehensive and user-friendly respiratory health platform.
\end{itemize}

\subsection*{Challenges and Constraints}
\begin{itemize}
    \item \textbf{Medical Accuracy:} Ensure the assistant provides reliable and clinically sound responses to avoid misinformation.
    
    \item \textbf{Responsiveness:} Maintain fast and natural conversational interactions to improve user experience.
    
    \item \textbf{Scope Limitation:} Focus the assistant strictly on respiratory health information, not on clinical diagnosis or treatment.
\end{itemize}