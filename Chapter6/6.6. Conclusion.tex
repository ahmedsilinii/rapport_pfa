\newpage


\section*{Conclusion}
\addcontentsline{toc}{section}{Conclusion}

\label{sec:system_summary}

This chapter provided a comprehensive overview of the system architecture and implementation details of the AI-powered mobile diagnosis application. The system is structured into three main layers: the mobile frontend, the backend infrastructure, and the AI inference engine. Each component was designed to support secure, efficient, and scalable workflows for users seeking preliminary medical feedback.

The mobile application, developed using Flutter, offers a cross-platform interface for users to describe symptoms and receive AI-generated diagnostic results. Its intuitive design and responsive UI enhance user engagement and streamline the diagnosis submission process.

The backend, powered by Appwrite, handles authentication, data storage, and serverless function execution. It acts as a secure intermediary between the mobile client and the AI inference engine, ensuring encrypted data handling, user-specific data isolation, and real-time result synchronization.

The AI pipeline, built using Python and transformer-based models, interprets audio symptom descriptions and returns structured diagnostic suggestions. It integrates seamlessly with the backend via REST APIs, enabling efficient model inference workflows.

Together, these components form a robust, modular architecture that supports accurate, secure, and real-time delivery of diagnostic information. The technology stack and integration strategy ensure future scalability, maintainability, and adaptability to more advanced diagnostic features.
