
\section*{Introduction}
\addcontentsline{toc}{section}{Introduction}

This chapter presents the implementation phase of our project: an AI-powered mobile application designed to analyze breathing sounds and assist in detecting respiratory diseases.

The implementation phase translates theoretical concepts into a working system. It focuses on the tools, technologies, and methodologies used to build the application and ensure its reliability and performance.

The core objective is to develop a mobile app capable of recording respiratory audio, uploading it securely, and using AI-based models to provide preliminary diagnostic feedback. The app also allows users to upload medical reports and offers a simple, intuitive interface with real-time insights.

To achieve this, we adopted a modular architecture that separates mobile development, backend services, and AI inference. The mobile app was developed using Flutter, chosen for its cross-platform capabilities and development speed. Appwrite was selected as the backend platform for its support of authentication, storage, and serverless functions.

This chapter begins with an overview of the system architecture and its components. It then details the implementation of key modules, including audio capture and encryption, data transmission, backend integration, and AI model deployment. Each section highlights the technical decisions made and the reasoning behind them.
