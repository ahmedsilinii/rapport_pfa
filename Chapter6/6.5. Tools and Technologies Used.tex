\section{Tools and Technologies Used}  
\label{sec:tools_tech}

The development of our AI-powered mobile diagnostic system was supported by a carefully selected suite of tools, frameworks, and platforms that together ensured efficiency, scalability, and security across the entire stack. This section introduces the core technologies employed throughout the project.

\subsection{Flutter for Cross-Platform Development}

Flutter was chosen as the core framework for building the mobile application, enabling a unified codebase for both Android and iOS platforms. Its declarative UI toolkit and reactive architecture allowed for efficient development and a smooth user experience.

The Flutter logo is shown in Figure~\ref{fig:flutter_logo}.

\begin{figure}[H]
    \centering
    \includegraphics[width=0.25\textwidth]{images/tools/flutter.png}
    \caption{Flutter Logo}
    \label{fig:flutter_logo}
\end{figure}

\subsection{Dart Programming Language}

Dart, the language behind Flutter, was used for implementing both application logic and UI elements. Its type safety and support for asynchronous operations made it well-suited for responsive mobile development.

The Dart logo is displayed in Figure~\ref{fig:dart_logo}.

\begin{figure}[H]
    \centering
    \includegraphics[width=0.2\textwidth]{images/tools/dart.png}
    \caption{Dart Logo}
    \label{fig:dart_logo}
\end{figure}

\subsection{Appwrite for Backend-as-a-Service}

Appwrite was selected as the backend-as-a-service solution, offering user authentication, database storage, file handling, and serverless functions. Its modular design simplified backend operations and enhanced security.

The Appwrite logo is presented in Figure~\ref{fig:appwrite_logo}.

\begin{figure}[H]
    \centering
    \includegraphics[width=0.3\textwidth]{images/tools/appwrite.png}
    \caption{Appwrite Logo}
    \label{fig:appwrite_logo}
\end{figure}

\subsection{Python for AI Model Deployment}

Python was used to develop and deploy the AI inference engine, leveraging libraries such as PyTorch and Hugging Face Transformers for audio-based diagnosis prediction.

The Python logo can be seen in Figure~\ref{fig:python_logo}.

\begin{figure}[H]
    \centering
    \includegraphics[width=0.25\textwidth]{images/tools/python.png}
    \caption{Python Logo}
    \label{fig:python_logo}
\end{figure}

\subsection{REST APIs for Communication}

RESTful APIs facilitated communication between the mobile frontend, Appwrite backend, and AI inference engine. This decoupled architecture allowed independent scaling and streamlined integration.

A representation of REST APIs is shown in Figure~\ref{fig:restapi_logo}.

\begin{figure}[H]
    \centering
    \includegraphics[width=0.2\textwidth]{images/tools/restapi.png}
    \caption{REST API Representation}
    \label{fig:restapi_logo}
\end{figure}

\subsection{Docker for Containerization}

Docker was employed to containerize both Appwrite services and the AI backend. This ensured consistent deployment, streamlined development, and enhanced scalability.

The Docker logo is illustrated in Figure~\ref{fig:docker_logo}.

\begin{figure}[H]
    \centering
    \includegraphics[width=0.3\textwidth]{images/tools/docker.png}
    \caption{Docker Logo}
    \label{fig:docker_logo}
\end{figure}

\subsection{VS Code and Postman for Development and Testing}

Visual Studio Code served as the primary development environment, providing extensive support for Dart, Flutter, and Python development.

The VS Code logo is shown in Figure~\ref{fig:vscode_logo}.

\begin{figure}[H]
    \centering
    \includegraphics[width=0.2\textwidth]{images/tools/vscode.png}
    \caption{VS Code Logo}
    \label{fig:vscode_logo}
\end{figure}

Postman was used for testing API endpoints and simulating request flows during diagnosis.

The Postman logo is displayed in Figure~\ref{fig:postman_logo}.

\begin{figure}[H]
    \centering
    \includegraphics[width=0.2\textwidth]{images/tools/postman.png}
    \caption{Postman Logo}
    \label{fig:postman_logo}
\end{figure}

\subsection{Android Studio for Android Emulation}

Android Studio was utilized to run the Android emulator, enabling thorough testing of the mobile application on virtual Android devices and ensuring compatibility and performance.

The Android Studio logo is shown in Figure~\ref{fig:androidstudio_logo}.

\begin{figure}[H]
    \centering
    \includegraphics[width=0.25\textwidth]{images/tools/androidstudio.png}
    \caption{Android Studio Logo}
    \label{fig:androidstudio_logo}
\end{figure}

\subsection{Git and GitHub for Version Control}

Version control was managed using Git, with GitHub as the remote repository. This setup supported collaborative development, issue tracking, and code versioning.

The Git and GitHub logos are presented in Figure~\ref{fig:git_github_logo}.

\begin{figure}[H]
    \centering
    \includegraphics[width=0.4\textwidth]{images/tools/git_github.png}
    \caption{Git and GitHub Logos}
    \label{fig:git_github_logo}
\end{figure}

\subsection{Summary}

The integration of these technologies ensured a robust, maintainable, and scalable solution. From frontend development to AI inference and deployment, each tool played a critical role in delivering a reliable diagnostic experience to end users.
