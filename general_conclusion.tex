\chapter*{Conclusion and Perspectives}
\addcontentsline{toc}{chapter}{Conclusion and Perspectives}
\rhead{Conclusion and Perspectives}

\section*{Conclusion}

This project presented the design and implementation of an AI-powered mobile application for breathing analysis and the detection of respiratory anomalies. The system combines mobile development with advanced AI techniques to deliver a modern, accessible diagnostic tool that supports early screening for conditions such as asthma, pneumonia, COPD, and sleep apnea.

The application allows users to record their breathing sounds, upload medical reports, and interact with an AI assistant. Through the use of Flutter for frontend development and Appwrite as a secure backend platform, the system ensures a smooth, cross-platform experience while maintaining user privacy and data integrity. On the AI side, the pipeline includes respiratory sound classification using deep learning models and personalized feedback powered by large language models (LLMs), enabling both automated diagnosis and natural language interaction.

The modular system architecture, built around RESTful APIs and containerized services, enabled seamless integration between the frontend, backend, and AI components. Tools like Docker, Git, Postman, and VS Code contributed to a robust and maintainable development workflow.

\section*{Perspectives}

While the current version of the application demonstrates the feasibility and effectiveness of AI-assisted respiratory analysis, several avenues remain for future improvement:

\begin{itemize}
  \item \textbf{Improved Audio Processing:} Incorporating advanced audio enhancement techniques, such as adaptive noise cancellation and real-time denoising, would increase the robustness of predictions in noisy environments.
  
  \item \textbf{Model Optimization and On-Device Inference:} Future iterations could explore model compression and quantization techniques to allow inference directly on the mobile device, improving response time and reducing backend dependency.

  \item \textbf{Dataset Expansion:} Increasing the diversity and size of the training dataset — including data from various age groups, devices, and environments — would improve the model’s generalization to real-world scenarios.

  \item \textbf{User Feedback Integration:} Allowing users to provide structured feedback on the accuracy of diagnoses and AI assistant responses could help continuously refine the system.

  \item \textbf{Medical Integration:} Long-term perspectives include integration with clinical workflows or partnerships with healthcare providers to support telehealth and digital triage systems.

  \item \textbf{Security and Regulatory Compliance:} Future versions should consider compliance with medical data regulations (e.g., GDPR, HIPAA) and formal security audits to ensure safe deployment in healthcare settings.
\end{itemize}

In conclusion, this project highlights the potential of combining mobile technologies with AI to support respiratory healthcare. With continued development and validation, this system can evolve into a trusted tool for early screening, patient monitoring, and AI-powered clinical assistance.
